\documentclass[a4paper,10pt]{article}
%\documentclass[10pt]{article}

%A Few Useful Packages
\usepackage{marvosym}
\usepackage{fontspec} 					%for loading fonts
\usepackage{xunicode,xltxtra,url,parskip} 	%other packages for formatting
\RequirePackage{color,graphicx}
\usepackage[usenames,dvipsnames]{xcolor}
\usepackage[big]{layaureo} 				%better formatting of the A4 page
%\usepackage{fullpage}
% an alternative to Layaureo can be ** \usepackage{fullpage} **
\usepackage{supertabular} 				%for Grades
\usepackage{titlesec}					%custom \section
\usepackage{multicol}
\usepackage{multirow}
\usepackage{longtable}
\usepackage{rotating}
\usepackage{ifthen}

\usepackage{hyperref}

\usepackage[absolute]{textpos}

\usepackage{enumitem}
%\renewcommand{\labelitemi}{\textcolor{lightg}{\symbol{"00BB}}}

% Prefixes bibtex output with "C" (for conference)
\makeatletter
\renewcommand*{\@biblabel}[1]{\hfill[C#1]}
\makeatother

%Setup hyperref package, and colours for links
\definecolor{linkcolour}{rgb}{0,0.2,0.6}
\hypersetup{colorlinks,breaklinks,urlcolor=linkcolour, linkcolor=linkcolour}

%Color
\definecolor{lightg}{HTML}{999999}
\definecolor{medg}{HTML}{666666}
\definecolor{darkg}{HTML}{333333}

% Bullets
\definecolor{noteone}{HTML}{999999}
\definecolor{notetwo}{HTML}{848484}
\definecolor{notethree}{HTML}{424242}
\definecolor{notefour}{HTML}{212121}
\definecolor{notefive}{HTML}{000000}

\newcommand{\fivenotes}{%
	\textcolor{noteone}{\symbol{"2022}}
	\textcolor{notetwo}{\symbol{"2022}}
	\textcolor{notethree}{\symbol{"2022}}
	\textcolor{notefour}{\symbol{"2022}}
	\textcolor{notefive}{\symbol{"2022}}
}
\newcommand{\fournotes}{%
	\textcolor{noteone}{\symbol{"2022}}
	\textcolor{notetwo}{\symbol{"2022}}
	\textcolor{notethree}{\symbol{"2022}}
	\textcolor{notefour}{\symbol{"2022}}
	\textcolor{white}{\symbol{"2022}}
}
\newcommand{\threenotes}{%
	\textcolor{noteone}{\symbol{"2022}}
	\textcolor{notetwo}{\symbol{"2022}}
	\textcolor{notethree}{\symbol{"2022}}
	\textcolor{white}{\symbol{"2022}}
	\textcolor{white}{\symbol{"2022}}
}
\newcommand{\twonotes}{%
	\textcolor{noteone}{\symbol{"2022}}
	\textcolor{notetwo}{\symbol{"2022}}
	\textcolor{white}{\symbol{"2022}}
	\textcolor{white}{\symbol{"2022}}
	\textcolor{white}{\symbol{"2022}}
}
\newcommand{\onenote}{%
	\textcolor{noteone}{\symbol{"2022}}
	\textcolor{white}{\symbol{"2022}}
	\textcolor{white}{\symbol{"2022}}
	\textcolor{white}{\symbol{"2022}}
	\textcolor{white}{\symbol{"2022}}
}

%FONTS
\defaultfontfeatures{Mapping=tex-text}
\setmainfont[SmallCapsFont = Fontin SmallCaps]{Fontin}

\font\lighttext=''Baskerville-Normal:color=787878'' at 10pt
\font\lighttextweb=''Baskerville-Normal:color=FF1493'' at 10pt

%CV Sections inspired by: 
%http://stefano.italians.nl/archives/26
\titleformat{\section}{\Large\scshape\raggedright}{}{0em}{}[\titlerule]
\titlespacing{\section}{0pt}{3pt}{3pt}
%Tweak a bit the top margin

%\addtolength{\voffset}{-1.3cm}

%-------------WATERMARK TEST [**not part of a CV**]---------------
\TPGrid[30mm,30mm]{30}{60}
%\setlength{\TPHorizModule}{30mm}
%\setlength{\TPVertModule}{\TPHorizModule}
%\textblockorigin{2mm}{0.65\paperheight}
\setlength{\parindent}{0pt}


%--------------------BEGIN DOCUMENT----------------------
\begin{document}

\pagestyle{empty} % non-numbered pages

\font\fb=''[cmr10]'' %for use with \LaTeX command

%--------------------TITLE-------------
\par{\centering
		{\Huge Jarrell \textsc{Waggoner}
	}\bigskip\par}

%--------------------SECTIONS-----------------------------------
%Section: Personal Data
\section{Biographical Data}

\begin{tabular}{r p{3.5in}}
	%\textsc{Birth:} & Lynchburg, VA \hspace{0.5em} | \hspace{0.5em} 24 September 1984 \\
	%\textsc{Address:}   & 241 S. Bull St. Columbia, SC 29205 \\
	\textsc{Address:}	& Department of Computer Science and Engineering, University of South Carolina, Columbia, SC 29208 \\
    \textsc{Phone:}     & 847-261-4747\\
    %\textsc{email:}     & \href{mailto:jarrell.waggoner@gmail.com}{jarrell.waggoner@gmail.com}
    \textsc{email:}     & \href{mailto:waggonej@email.sc.edu}{waggonej@email.sc.edu} \\
	\textsc{Website:}	& \href{http://www.malloc47.com}{www.malloc47.com} \\
	\textsc{Citizenship:} & United States Citizen \\
\end{tabular}

\section{Research Interests}

Computer vision, segmentation, contour completion, perceptural grouping, document image analysis, event recognition, image processing, signal processing, tomographic image analysis.

\section{Education}
\begin{tabular}{r p{12cm}}	
 \textsc{Present} & Ph.D. Candidate in \textsc{Computer Science}, \textbf{University of South Carolina}\\
&% Dissertation: ``Sublinear and Locally Sublinear Prices'' |
\small Advisor: Dr. Song \textsc{Wang} % | \normalsize \textsc{Gpa}: 3.85/4.0
%\hyperlink{grds}{\hfill | \footnotesize Detailed List of Exams}
\\&\\
 \textsc{May} 2009 & Master of Engineering in \textsc{Computer Science}, \textbf{University of South Carolina}\\
&\normalsize \textsc{GPA}: 3.8/4.0 | \small\emph{magna cum laude}
%\hyperlink{grds}{\hfill | \footnotesize Detailed List of Exams}
\\&\\
\textsc{May} 2006& Bachelor of Science in \textsc{Computer Science}, \textbf{Bryan College} \\
%& Thesis: ``Aspect-Oriented Programming in an Extreme Programming Environment''\\
&\normalsize 
%\textsc{GPA}: ?.?/4.0 | 
\small\emph{summa cum laude}
%\hyperlink{grds_cleli}{\hfill| \footnotesize Detailed List of Exams}
\\&\\
\textsc{May} 2004& Associate of Science in \textsc{Computer Science}\\
&\textbf{University of South Carolina at Lancaster} \\
&\normalsize \textsc{GPA}: 4.0/4.0 | \small\emph{summa cum laude}\\
%\hyperlink{grds_cleli}{\hfill| \footnotesize Detailed List of Exams}
\end{tabular}

\section{Teaching Experience}
\begin{longtable}{r|p{11cm}}
%\emph{Current} 
\textsc{2008---2009}
& GK-12 Fellow at \textsc{Crayton Middle School} \\
%in conjunction with the \textsc{Center for Teaching Excellence} \\
&\emph{Teaching 8\textsuperscript{th} Grade Science}\\
&\footnotesize{Served in Crayton Middle School, coordinating with the classroom instructor to enhance the science curriculum and activities in an 8\textsuperscript{th} grade science classroom. Subsequently coordinated and taught at the \textsc{GK-12 Institute for Teachers}, presenting the activities developed and delivered in the classroom.}
\\\multicolumn{2}{c}{} \\
\textsc{2007---2008} & Graduate Teaching Assistant at \textsc{USC} \\
&\emph{Teaching Software Development and Web Scripting}\\
&\footnotesize{Supervised CSCE~145 labs, covering software development with \textsc{Java}, and taught CSCE~102, covering \textsc{Javascript}, \textsc{HTML}, and \textsc{CSS}.}\\\multicolumn{2}{c}{} \\
\textsc{Spring 2007} &  Instructor for \textsc{CSCE 204} at \textsc{USCL} \\
&\emph{Teaching Introductory Programming}\\
&\footnotesize{Hired as special faculty. Taught introductory Visual Basic for majors and non-majors. Selected textbooks, developed all course material, graded all assignments. Worked with Dr. Noni M. Bohonak}\\\multicolumn{2}{c}{} \\
\textsc{Fall 2006} & Camp Instructor for \textsc{USCL Arts and Sciences Adventure Camp} \\
&\emph{Teaching 5\textsuperscript{th}-8\textsuperscript{th} Grade Students}\\
&\footnotesize{Worked in collaboration with Dr. Dwayne Brown. One of two instructors teaching Math and Computer Science to grade school students.}\\\multicolumn{2}{c}{} \\
\textsc{2003---2007} & Professional Tutor at \textsc{USCL Academic Success Center} \\
&\emph{Tutoring High School and College Students}\\
&\footnotesize{Student and graduate tutor for college-level Mathematics, Computer Science, Physics, and English classes.}\\\multicolumn{2}{c}{} \\
\end{longtable}


\section{Research Experience}
\begin{tabular}{r|p{11cm}}
%\emph{Current} 
\textsc{2010---Present}
& Research Assistant funded by \textsc{DARPA} \\
&\emph{Video Event Recognition}\\
&\footnotesize{Explored segmentation methods for video event recognition while working at the \textsc{Computer Vision Lab} at \textsc{USC}. Managed lab computer network and organize weekly lab meetings.  Attended P.I. meetings in San Diego (2010) and Colorado (2011). Visited \textsc{Purdue University} working with Dr.~Jeffrey Mark Siskind (Dec~2010---Jan~2011).}
\\\multicolumn{2}{c}{} \\
\textsc{2009---2010}
& NSF Fellow at the \textsc{Center for Digital Humanities} \\
&\emph{Digital Collation}\\
&\footnotesize{Created a \textsc{digital collation} application to handle automatic differencing of sub-textual inconsistencies among multiple copies of \emph{The Faerie Queene} by \textsc{Edmund Spenser}.}
\\\multicolumn{2}{c}{} \\
\end{tabular}


%\section{Publications}
\nocite{temlyakov2010}
\nocite{zhang2010}
\renewcommand\refname{Publications}
\bibliography{cv}
\bibliographystyle{plainyr-rev}


\section{Presentations}
\begin{enumerate}
\renewcommand{\labelenumi}{[P\arabic{enumi}] }
\item \emph{Image Registration for Digital Collation}. Graduate Student Day Competition, Honorable Mention. April 2, 2010.
\item \emph{Aspect-Oriented Programming}. In CSCE 531. Guest lecture for Dr. Marco Valtorta. March 19, 2008.
\item \emph{Math 241}. Vector Calculus. Guest lecture for Dr. Dwayne Brown. April~23---26, 2007. 
\item \emph{Math 242}. Differential Equations. Guest lecture for Dr. Dwayne Brown. April~23---26, 2007. 
\end{enumerate}

\section{Travel}
\begin{enumerate}
\renewcommand{\labelenumi}{[T\arabic{enumi}] }
\item \emph{Mind's Eye PI Meeting}. DARPA Project P.I.~meeting. Denver, CO. January 20---21, 2011. 
\item \emph{Tomography and its Applications to Materials Science and Non-Destructive Evaluation}. Organizd by M. De Graef, L. Drummy, J. Simmons, M. Comer, C. Bouman, and J. Knopp. Tech\^{}Edge, Dayton, Ohio. December 13---15, 2010.
\item Visiting scholar. In collaboration with J. M. Siskind. Purdue University, West Lafayette, IN. December 6---22, 2010 \& January 5---16, 2011.
\item \emph{Mind's Eye Kickoff Meeting}. DARPA Project P.I.~meeting. San Diego, CA. September 23---24, 2010. 
\end{enumerate}


\section{Honors/Awards}
\begin{tabular}{rll}
2010 & Graduate Student Day Presentation,  Honorable Mention & {\lighttext \textcolor{lightg}{USC}}\\
2006 & Senior Computer Science Award & {\lighttext \textcolor{lightg}{Bryan College}}\\
2004 & Clara P. Hammond Award & \multirow{3}{*}{{\lighttext \textcolor{lightg}{\begin{turn}{-90}USCL\end{turn}}}} \\
& Science and Mathematics Award \\
& Highest Academic Average Award \\
\end{tabular}

%\section{Professional Societies}

\section{Teaching}
\begin{center}
%\begin{tabular*}{0.75\textwidth}{r @{\hspace{0.5em}\textcolor{lightg}{\symbol{"00BB}}\hspace{0.5em}} l @{\extracolsep{\fill}} l }
\begin{tabular*}{0.75\textwidth}{r @{\hspace{0.5em}\textcolor{lightg}{\symbol{"00BB}}\hspace{0.5em}} l l c }
%\multicolumn{3}{r}{University of South Carolina}\\
Summer II 2008 & CSCE 102 & HTML/CSS/Javasript & \multirow{3}{*}{{\lighttext \textcolor{lightg}{\begin{turn}{-90}USC\end{turn}}}} \\
Spring 2008 & CSCE 145 Lab & Java \\
Fall 2007 & CSCE 145 Lab & Java \\
\multicolumn{3}{r}{}\\
%\multicolumn{3}{r}{University of South Carolina at Lancaster}\\ \hline
Spring 2007 & CSCE 204 & Visual Basic & \multirow{3}{*}{{\lighttext \textcolor{lightg}{\begin{turn}{-90}USCL\end{turn}}}} \\
Spring 2007 & Math 241 \& Math 242 & Maple \\
\multicolumn{1}{c}{}& (Guest Lecture) & \\
\end{tabular*}
\end{center}

\section{Skills \& Languages}
\begin{multicols}{3}
\raggedcolumns
%\parbox{5cm}{
%\begin{enumerate}
\begin{itemize}
%\renewcommand{\labelenumi}{\textcolor{lightg}{[S\arabic{enumi}]}}
\renewcommand{\labelitemi}{\textcolor{lightg}{\symbol{"00BB}}}
\setlength{\itemsep}{1pt}
\setlength{\parskip}{0pt}
\setlength{\parsep}{0pt}
\item Assembly \hfill \twonotes
\item Bash \hfill \fournotes
\item Blender \hfill \threenotes
\item C/C++ \hfill \fournotes 
\item English \hfill \fivenotes
\item GIT/SVN/CVS \hfill \fournotes
\item GNU/Linux \hfill \fournotes
\item HTML/CSS \hfill \fivenotes
%\item Image Processing \hfill \fivenotes
\item Java \hfill \fivenotes
\item Javacript \hfill \threenotes
%\item LAMP Stack \hfill \fournotes
\item \LaTeX \hfill \threenotes
\item LISP \hfill \onenote
%\item Machine Learning \hfill \fournotes
\item Maple \hfill \threenotes
\item MATLAB \hfill \fivenotes
%\item MS Office \hfill \fivenotes
%\item Networking \hfill \threenotes
\item OpenCV \hfill \fournotes
\item PHP \hfill \fournotes
\item Python \hfill \threenotes
\item Scheme \hfill \threenotes
\item SQL \hfill \threenotes
\item Sys. Admin. \hfill \fournotes
\item Visual Basic \hfill \fivenotes
%\item Windows \hfill \fivenotes
\item Wordpress \hfill \fournotes
%\end{enumerate}
\end{itemize}
%}
\end{multicols}

\section{Interests and Activities}
Programming, Teaching, Mathematics\\
Open-source Software, System Administration, Linux\\
Typography, Music Composition

%%\XeTeXpdffile ''GMAT.pdf'' page 1 scaled 800

\end{document}
