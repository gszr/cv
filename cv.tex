\documentclass[a4paper,10pt]{article}

\usepackage{fontspec}
\usepackage[usenames,dvipsnames]{xcolor}
\usepackage[big]{layaureo}
\usepackage{supertabular}
\usepackage{titlesec}
\usepackage{multicol}
\usepackage{multirow}
\usepackage{longtable}
\usepackage{rotating}
\usepackage{ifthen}
\usepackage{hyperref}
\usepackage[absolute]{textpos}
\usepackage{enumitem}
\usepackage[normalem]{ulem}

\usepackage{geometry}
\geometry{
  noheadfoot=true,
  top=1.5cm,
  bottom=1.5cm
}

\pagestyle{empty}

%Setup hyperref package, and colours for links
\definecolor{linkcolour}{rgb}{0,0.2,0.6}
\hypersetup{colorlinks,breaklinks,urlcolor=linkcolour, linkcolor=linkcolour}

%Color
\definecolor{lightg}{HTML}{999999}
\definecolor{medg}{HTML}{666666}
\definecolor{darkg}{HTML}{333333}

% Bullets
\definecolor{noteone}{HTML}{999999}
\definecolor{notetwo}{HTML}{848484}
\definecolor{notethree}{HTML}{424242}
\definecolor{notefour}{HTML}{212121}
\definecolor{notefive}{HTML}{000000}

\newcommand{\fivenotes}{%
	\textcolor{noteone}{\symbol{"2022}}
	\textcolor{notetwo}{\symbol{"2022}}
	\textcolor{notethree}{\symbol{"2022}}
	\textcolor{notefour}{\symbol{"2022}}
	\textcolor{notefive}{\symbol{"2022}}
}
\newcommand{\fournotes}{%
	\textcolor{noteone}{\symbol{"2022}}
	\textcolor{notetwo}{\symbol{"2022}}
	\textcolor{notethree}{\symbol{"2022}}
	\textcolor{notefour}{\symbol{"2022}}
	\textcolor{white}{\symbol{"2022}}
}
\newcommand{\threenotes}{%
	\textcolor{noteone}{\symbol{"2022}}
	\textcolor{notetwo}{\symbol{"2022}}
	\textcolor{notethree}{\symbol{"2022}}
	\textcolor{white}{\symbol{"2022}}
	\textcolor{white}{\symbol{"2022}}
}
\newcommand{\twonotes}{%
	\textcolor{noteone}{\symbol{"2022}}
	\textcolor{notetwo}{\symbol{"2022}}
	\textcolor{white}{\symbol{"2022}}
	\textcolor{white}{\symbol{"2022}}
	\textcolor{white}{\symbol{"2022}}
}
\newcommand{\onenote}{%
	\textcolor{noteone}{\symbol{"2022}}
	\textcolor{white}{\symbol{"2022}}
	\textcolor{white}{\symbol{"2022}}
	\textcolor{white}{\symbol{"2022}}
	\textcolor{white}{\symbol{"2022}}
}

%FONTS
\defaultfontfeatures{Mapping=tex-text}
\setmainfont[SmallCapsFont = Fontin SmallCaps]{Fontin}

\font\lighttext=''Baskerville-Normal:color=787878'' at 10pt
\font\lighttextweb=''Baskerville-Normal:color=FF1493'' at 10pt

%CV Sections inspired by: 
%http://stefano.italians.nl/archives/26
\titleformat{\section}{\Large\scshape\raggedright}{}{0em}{}[\titlerule]
\titlespacing{\section}{0pt}{3pt}{3pt}
%Tweak a bit the top margin

%\addtolength{\voffset}{-1.3cm}

%-------------WATERMARK TEST [**not part of a CV**]---------------
\TPGrid[30mm,30mm]{30}{60}
%\setlength{\TPHorizModule}{30mm}
%\setlength{\TPVertModule}{\TPHorizModule}
%\textblockorigin{2mm}{0.65\paperheight}
\setlength{\parindent}{0pt}


\begin{document}

\font\fb=''[cmr10]''

\center{\huge\uline{{Guilherme Salazar}}}
\center{\large{Goiânia, Brazil} \\
\href{mailto:gmesalazar@acm.org}{gsz@acm.org} \\
\href{https://www.github.com/salazar/}{github.com/salazar}\vspace{3ex}
}

\section{Education}

\begin{tabular}{r|p{11cm}}

  \textsc{Jan '10--Jul '16} & Computer Science, B.Sc\\ &
  \emph{Universidade Federal de Goiás, Brazil} \\ &
  \footnotesize{First year and a half were as a Software Engineering major}
  \\\multicolumn{2}{c}{} \\

  \textsc{Aug '12--Jul '13} & Computer Science, Non-degree Program \\ &
  \emph{University of South Carolina, United States} \\ &
  \footnotesize{Scholarship granted by the Brazilian Government} \\ &
  \footnotesize{Dean Honor List \& Alpha Lambda Delta}
  \\\multicolumn{2}{c}{} \\
 
\end{tabular}

\section{Experience}

\setlength{\LTpre}{0pt}
\setlength{\LTpost}{0pt}

\begin{longtable}{r|p{11cm}}

  \textsc{Jul '17--Cur} & Software Engineer\\
  &\emph{\href{https://konghq.com}{Kong Inc.}}\\
  &\footnotesize{Part of the team that develops and maintains the Enterprise
  offering of \href{https://konghq.com/}{Kong}, the world's most popular
  open-source API gateway and microservices management layer.}
  \\\multicolumn{2}{c}{} \\

  \textsc{Nov '16--Jul '17} & Software Engineer\\
  &\emph{\href{https://appbase.io}{Appbase, Inc}}\\
  &\footnotesize{Building real-time abstractions on top of
  Elasticsearch; the product is based on OpenResty (Nginx + Lua) and Docker.}
  \\\multicolumn{2}{c}{} \\

  \textsc{Dec '15--Cur} & Developer\\ &\emph{The NetBSD
  Foundation}\\ &\footnotesize{Maintainer for Lua user and kernel
  space implementations, as well as the test suite for kernel Lua.
  I've done work porting parts of Lua standard
  libraries to kernel space--e.g., \emph{base, math, io, os} and
  developing I/O bindings to file systems and sockets.
  (\href{https://GitHub.com/salazar/netbsd}{GitHub.com/salazar/netbsd}})
  \\\multicolumn{2}{c}{} \\

  \textsc{Oct '15--Feb '16} & Software Engineering Intern \\ &\emph{Zenedge,
  LTDA}\\ \multicolumn{2}{c}{} \\

  \textsc{Apr--Aug '15} & Student Developer \\ &\emph{Google Summer of
  Code, LabLua}\\ &\footnotesize{Ported Lua test suite to NetBSD
  kernel Lua; tests
  had to be adapted to be free offloating-point numbers and parts of Lua's standard libraries had to be
  ported using kernel interfaces.
  (\href{https://GitHub.com/salazar/luatests}{GitHub.com/salazar/luatests}})
  \\\multicolumn{2}{c}{} \\

  \textsc{Jan '14--Jul '15} & Programmer \\ &\emph{Macro Softwares,
  LTDA}\\ &\footnotesize{Developed an Android app for package delivery
  management along with Spring RESTful web services used by the app
  and related software. It's on Google Play under the name "Entregador
  Online"}\\ \multicolumn{2}{c}{} \\

  \textsc{Jan--Jul '13} & Programmer \\ &\emph{University of South Carolina}\\
  &\footnotesize{Developed a web app with electronic health
  records (EHR) features using Google App Engine
  (Python), along with Bootstrap and jQuery on the front end.
  Under Dr. \href{http://jmvidal.cse.sc.edu}{José Vidal}}\\

  \multicolumn{2}{c}{} \\

  \textsc{2011, 2012, 2014, 2015} & Teaching \& Research Assistant \\ &\emph{Instituto
  de Informática, Universidade Federal de Goiás} \\
  &\footnotesize{$\circ$ Sep '11--Jul '12: Smart Cities (IoT): Study, experimentation,
  and Evaluation of IP stack implementations for low-power devices (e.g., TelosB
  and MicaZ motes) -- under Dr. \href{http://inf.ufg.br/~brunoos/}{Bruno Silvestre}} \\
  &\footnotesize{$\circ$ Mar--Sep '11: Concurrent \& Distributed Systems: Study of
  models and mechanisms of coordination and communication in
  event-driven concurrent and distributed systems under Dr.
  \href{http://inf.ufg.br/~brunoos/}{Bruno Silvestre}} \\
  &\footnotesize{$\circ$ Mar--Jul '15: Formal Languages and Automata} \\
  &\footnotesize{$\circ$ Mar--Jul '14: Operating Systems} \\
  &\footnotesize{$\circ$ Aug--Dec '11: Software Construction} \\

\end{longtable}

\section{Skills}

\begin{itemize}
\renewcommand{\labelitemi}{\textcolor{lightg}{\symbol{"00BB}}}
\setlength{\itemsep}{1pt}
\setlength{\parskip}{0pt}
\setlength{\parsep}{0pt}

\begin{multicols}{3}
  \item Portuguese (BR) \hfill \threenotes
  \item English (US)    \hfill \threenotes
  \item C          \hfill \threenotes 
  \item Lua        \hfill \threenotes
  \item Python     \hfill \twonotes
  \item Java       \hfill \twonotes
  \item JavaScript \hfill \onenote
  \item Shell Scripting \hfill \twonotes
  \item Go         \hfill \onenote
  \item nginx         \hfill \twonotes
  \item OpenResty     \hfill \twonotes
  \item Docker        \hfill \twonotes
  \item AWS (EC2)     \hfill \onenote
  \item Elasticsearch \hfill \onenote
  \item Redis         \hfill \onenote
  \item SQL           \hfill \onenote
  \item GNU/Linux \hfill \threenotes
  \item FreeBSD   \hfill \twonotes
  \item NetBSD    \hfill \twonotes
  \item Android   \hfill \onenote
  \item Vagrant \hfill \onenote
  \item Git     \hfill \twonotes
  \item SVN     \hfill \onenote
  \item CVS     \hfill \onenote
\end{multicols}

\end{itemize}

\vspace{0.5em}

\begin{center}
\parbox[c]{8cm}{
  \onenote Familiar; small-scale projects \\
  \twonotes Quite familiar; used in larger projects \\
  \threenotes Extensive knowledge
}
\end{center}

\section{Courses}

\begin{tabular}{r|p{11cm}}

  \textsc{Jul--Aug '12} & Summer Pre-Academic Program\\ &
  \emph{CHHS, University of North Carolina, Charlotte}\\ & 115 hours
  \\\multicolumn{2}{c}{} \\

  \textsc{Jan '10--Jul '11} & English Language Course\\ &
  \emph{Cultural Norte-Americano} \\ & 120 hours\\

\end{tabular}

\section{Publications}
\begin{enumerate}
  \renewcommand{\labelenumi}{[\arabic{enumi}] }
  \item Bruno Silvestre, Guilherme Salazar. \emph{Proposta de um Modelo para o
        Processamento de Eventos Concorrentes no ALua.} Encontro Anual de
        Computação, Universidade Federal de Goiás. October 26, 2011 (best paper
        award)
\end{enumerate}

\section{Associations}

\begin{tabular}{r|p{11cm}}

  \textsc{Apr '18--Cur}
  & USENIX Advanced Computing Systems Association \\
  \multicolumn{2}{c}{}\\

  \textsc{May '16--May '17}
  & Brazilian Computing Society\\
  \multicolumn{2}{c}{}\\

  \textsc{Dec '11--Cur}
  & Association for Computing Machinery\\

\end{tabular}

\section{Awards}
\begin{tabular}{r|p{11cm}}

  \textsc{2012--2013}
  & Dean's Honor List\\
  & College of Engineering \& Computing, University of South Carolina\\
  \multicolumn{2}{c}{}\\

  \textsc{2012}
  & Study Abroad Scholarship\\
  & Ministry of Education, Government of Brazil\\
  \multicolumn{2}{c}{}\\

  \textsc{2011}
  & Best Paper Award\\
  & Encontro Anual de Computação, Universidade Federal de Goiás\\

\end{tabular}

\end{document}
