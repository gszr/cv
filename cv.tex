\documentclass[a4paper,10pt]{article}

\usepackage{fontspec}
\RequirePackage{color,graphicx}
\usepackage[usenames,dvipsnames]{xcolor}
\usepackage[big]{layaureo}
\usepackage{supertabular}
\usepackage{titlesec}
\usepackage{multicol}
\usepackage{multirow}
\usepackage{longtable}
\usepackage{rotating}
\usepackage{ifthen}
\usepackage{hyperref}
\usepackage[absolute]{textpos}
\usepackage{enumitem}
\usepackage[normalem]{ulem}
\usepackage{calc}

\pagestyle{empty}

%Setup hyperref package, and colours for links
\definecolor{linkcolour}{rgb}{0,0.2,0.6}
\hypersetup{colorlinks,breaklinks,urlcolor=linkcolour, linkcolor=linkcolour}

%Color
\definecolor{lightg}{HTML}{999999}
\definecolor{medg}{HTML}{666666}
\definecolor{darkg}{HTML}{333333}

% Bullets
\definecolor{noteone}{HTML}{999999}
\definecolor{notetwo}{HTML}{848484}
\definecolor{notethree}{HTML}{424242}
\definecolor{notefour}{HTML}{212121}
\definecolor{notefive}{HTML}{000000}

\newcommand{\fivenotes}{%
	\textcolor{noteone}{\symbol{"2022}}
	\textcolor{notetwo}{\symbol{"2022}}
	\textcolor{notethree}{\symbol{"2022}}
	\textcolor{notefour}{\symbol{"2022}}
	\textcolor{notefive}{\symbol{"2022}}
}
\newcommand{\fournotes}{%
	\textcolor{noteone}{\symbol{"2022}}
	\textcolor{notetwo}{\symbol{"2022}}
	\textcolor{notethree}{\symbol{"2022}}
	\textcolor{notefour}{\symbol{"2022}}
	\textcolor{white}{\symbol{"2022}}
}
\newcommand{\threenotes}{%
	\textcolor{noteone}{\symbol{"2022}}
	\textcolor{notetwo}{\symbol{"2022}}
	\textcolor{notethree}{\symbol{"2022}}
	\textcolor{white}{\symbol{"2022}}
	\textcolor{white}{\symbol{"2022}}
}
\newcommand{\twonotes}{%
	\textcolor{noteone}{\symbol{"2022}}
	\textcolor{notetwo}{\symbol{"2022}}
	\textcolor{white}{\symbol{"2022}}
	\textcolor{white}{\symbol{"2022}}
	\textcolor{white}{\symbol{"2022}}
}
\newcommand{\onenote}{%
	\textcolor{noteone}{\symbol{"2022}}
	\textcolor{white}{\symbol{"2022}}
	\textcolor{white}{\symbol{"2022}}
	\textcolor{white}{\symbol{"2022}}
	\textcolor{white}{\symbol{"2022}}
}

%FONTS
\defaultfontfeatures{Mapping=tex-text}
\setmainfont[SmallCapsFont = Fontin SmallCaps]{Fontin}

\font\lighttext=''Baskerville-Normal:color=787878'' at 10pt
\font\lighttextweb=''Baskerville-Normal:color=FF1493'' at 10pt

%CV Sections inspired by: 
%http://stefano.italians.nl/archives/26
\titleformat{\section}{\Large\scshape\raggedright}{}{0em}{}[\titlerule]
\titlespacing{\section}{0pt}{3pt}{3pt}
%Tweak a bit the top margin

%\addtolength{\voffset}{-1.3cm}

%-------------WATERMARK TEST [**not part of a CV**]---------------
\TPGrid[30mm,30mm]{30}{60}
%\setlength{\TPHorizModule}{30mm}
%\setlength{\TPVertModule}{\TPHorizModule}
%\textblockorigin{2mm}{0.65\paperheight}
\setlength{\parindent}{0pt}


\begin{document}

\font\fb=''[cmr10]''

\par{\center{\Huge\uline{{Guilherme Salazar}}}\bigskip\par\vspace{8ex}}

\section{Personal}
\begin{tabular}{l@{\hskip 0.2cm}l}
  Born & \texttt{on} Sep 13, '91 \texttt{in}
                     \href{https://en.wikipedia.org/wiki/Goiania}{Goiânia}, BR \\
  Code & \texttt{at} \href{https://www.github.com/salazar/}
                          {GitHub.com/salazar} \\
  Page & \texttt{at} \href{https://www.netbsd.org/~salazar}
                          {NetBSD.org/\textasciitilde salazar} \\
  Mail & \texttt{at} \href{mailto:salazar@NetBSD.org}
                          {salazar@NetBSD.org} 
\end{tabular}

\section{Education}
\begin{tabular}{r|p{11cm}}

  \textsc{Aug '12--Jul '13} & Computer Science, Visiting\\ &
  \emph{College of Engineering \& Computing}\\ &
  \textbf{University of South Carolina, United States}
  \\\multicolumn{2}{c}{} \\

  \textsc{Aug '11--Jul '16} & Computer Science, B.Sc\\ &
  \emph{Instituto de Informática}\\ &
  \textbf{Universidade Federal de Goiás, Brazil}
  \\\multicolumn{2}{c}{} \\

  \textsc{Mar '10--Aug '11} & Software Engineering, Dropped \\ &
  \emph{Instituto de Informática}\\ &
  \textbf{Universidade Federal de Goiás, Brazil}
  \\\multicolumn{2}{c}{}

\end{tabular}

\section{Courses}
\begin{tabular}{r|p{11cm}}

  \textsc{Jul--Aug '12} & Summer Pre-Academic Program\\ &
  \emph{College of Health \& Human Services}\\ &
  \emph{University of North Carolina}\\ &
  Hour load: 115 hours
  \\\multicolumn{2}{c}{} \\

  \textsc{Jan '10--Jul '11} & English Language Course\\ &
  \emph{Cultural Norte-Americano} \\ &
  Hour load: 120 hours
  \\\multicolumn{2}{c}{} \\

  \textsc{Oct '10} & Aggregation of Software Engineering Experiments\\ &
  \emph{Instituto de Informática} \\ &
  \emph{Universidade Federal de Goiás}\\ &
  Hour load: 8 hours
  \\\multicolumn{2}{c}{} \\

  \textsc{Oct '10} & Introduction to Metrics and Measurement of Software\\ &
  \emph{Instituto de Informática} \\ &
  \emph{Universidade Federal de Goiás}\\ &
  Hour load: 8 hours
  \\\multicolumn{2}{c}{} \\

  \textsc{Oct '10} & Parallel Programming in CUDA\\ &
  \emph{Instituto de Informática} \\ &
  \emph{Universidade Federal de Goiás}\\ &
  Hour load: 10 hours
  \\\multicolumn{2}{c}{} \\

\end{tabular}

\section{Work}
\begin{tabular}{r|p{11cm}}

  \textsc{Dec '15--Cur}
  & Programmer \\
  &\emph{The NetBSD Foundation}\\
  &\emph{\footnotesize{Working with kernel Lua, the port of the Lua language to
          kernel space; currently, aims to bring user level Lua standard liraries 
          to kernel space and help developing new scenarios of use for kernel 
          Lua}}
  \\\multicolumn{2}{c}{} \\
  
  \textsc{May--Aug '15}
  & Programmer \\
  &\emph{Google Summer of Code, Google Inc., LabLua}\\
  &\emph{\footnotesize{\href{https://www.google-melange.com/gsoc/project/details/google/gsoc2015/gmesalazar/5741031244955648}{Port Lua test suite to the NetBSD kernel}}}\\
  &\emph{Under Lourival Neto}
  \\\multicolumn{2}{c}{} \\

  \textsc{Mar--Jul '15}
  & Teaching Assistant\\
  &\emph{Instituto de nformática, Universidade Federal de Goiás}\\
  &\emph{\footnotesize{Formal Languages and Automata Theory}} \\
  &\emph{Under Dr. {Márcia Cappelle}} 
  \\\multicolumn{2}{c}{} \\

  \textsc{Mar--Jul '14}
  & Teaching Assistant \\
  &\emph{Instituto de Informática, Universidade Federal de Goiás}\\
  &\emph{\footnotesize{Operating Systems}} \\
  &\emph{Under Dr. \href{http://inf.ufg.br/~ricardo/}{Ricardo Rocha}}
  \\\multicolumn{2}{c}{} \\

  \textsc{Jan--Jul '13}
  & Programmer \\
  &\emph{College of Engineering \& Computing, University of South Carolina}\\
  &\footnotesize{Development of a GAE app for USC's Medical School}\\
  &\emph{Under Dr. \href{http://jmvidal.cse.sc.edu}{José Vidal}}
  \\\multicolumn{2}{c}{} \\

  \textsc{Sep '11--Jul '12}
  & Research Assistant \\
  &\emph{Rede Nacional de Ensino e Pesquisa}\\
  &\footnotesize{Study, experimentation, and Evaluation of IP stack implementations 
  for low-power devices (e.g., TelosB and MicaZ motes). See 
            \href{http://www.nr2.ufpr.br/cia2/}{the page} of the main project}\\
  &\emph{Under Dr. \href{http://inf.ufg.br/~brunoos/}{Bruno Silvestre}}
  \\\multicolumn{2}{c}{} \\

  \textsc{Aug--Dec '11}
  & Teaching Assistant \\
  &\emph{Instituto de Informática, Universidade Federal de Goiás}\\
  &\emph{\footnotesize{Software Construction}} \\
  &\emph{Under Dr. Fabbrizio Soares}
  \\\multicolumn{2}{c}{} \\

  \textsc{Mar--Sep '11}
  & Research Assistant \\
  &\emph{Instituto de Informática, Universidade Federal de Goiás}\\
  &\emph{Supporting the Development of Concurrent and Distributed Systems}\\
  &\indent \footnotesize{Study of models and mechanisms of coordination and 
        communication in event-driven concurrent and distributed systems; the 
        use of Lua in the development of higher-level concurrency abstractions.}\\
  &\emph{Under Dr. \href{http://inf.ufg.br/~brunoos/}{Bruno Silvestre}}
  \\\multicolumn{2}{c}{} \\

\end{tabular}

\section{Publications}
\begin{enumerate}
  \renewcommand{\labelenumi}{[\arabic{enumi}] }
  \item Bruno Silvestre, Guilherme Salazar. Proposta de um Modelo para o
        Processamento de Eventos Concorrentes no ALua. Encontro Anual de
        Computação, Universidade Federal de Goiás. October 26, 2011 \\
\end{enumerate}

\section{Associations}

\begin{tabular}{r|p{11cm}}

  \textsc{Dec '15--Cur}
  & The NetBSD Foundation\\
  & \footnotesize{\url{http://www.netbsd.org/~salazar}}
  \\\multicolumn{2}{c}{}\\

  \textsc{Feb '13--Cur}
  & Alpha Lambda Delta Honor Society\\
  & \footnotesize{Iniciated at the University of South Carolina, Columbia, on February 11, 2013}\\
  \multicolumn{2}{c}{}\\

  \textsc{Dec '11--Cur}
  & Association for Computing Machinery\\
  & \footnotesize{\url{http://member.acm.org/~gmesalazar}}
  \\\multicolumn{2}{c}{}\\
  
\end{tabular}

\section{Awards}
\begin{tabular}{r|p{11cm}}

  \textsc{'12--'13}
  & Dean's Honor List\\
  & College of Engineering \& Computing, University of South Carolina\\
  \multicolumn{2}{c}{}\\

  \textsc{'12}
  & Study Abroad Scholarship\\
  & Ministry of Education, Government of Brazil\\
  \multicolumn{2}{c}{}\\

  \textsc{'11}
  & Best Paper Award\\
  & Encontro Anual de Computação, Universidade Federal de Goiás\\
  \multicolumn{2}{c}{}\\

\end{tabular}

\section{Skills}

\begin{multicols}{3}
  \begin{itemize}
    \renewcommand{\labelitemi}{\textcolor{lightg}{\symbol{"00BB}}}
    \setlength{\itemsep}{1pt}
    \setlength{\parskip}{0pt}
    \setlength{\parsep}{0pt}
  \item C \hfill \threenotes 
  \item Java \hfill \threenotes
  \item Lua \hfill \threenotes
  \item nesC \hfill \onenote
  \item Python \hfill \twonotes
  \item JavaScript \hfill \twonotes
  \item HTML/CSS \hfill \twonotes
  \item \LaTeX \hfill \twonotes
  \item SQL \hfill \twonotes
  \item Android \hfill \twonotes
  \item GAE \hfill \twonotes
  \item GNU/Linux \hfill \threenotes
  \item NetBSD \hfill \twonotes
  \item Windows \hfill \onenote
  \item Vagrant \hfill \twonotes
  \item Git \hfill \twonotes
  \item GitHub \hfill \twonotes
  \item SVN \hfill \onenote
  \item CVS \hfill \onenote
  \item BR Portuguese \hfill \twonotes
  \item US English \hfill \twonotes
  \end{itemize} 
\end{multicols}

\vspace{1em}

\begin{center}
\parbox[c]{8cm}{
  \onenote Familiar; small-scale projects \\
  \twonotes Quite familiar; used in large projects \\
  \threenotes Extensive knowledge
}
\end{center}

\section{Misc}
\centering{A lot of stuff interest me. Open-source software is one of them.
I believe openness and freedom---not only in software---can lead to a better world. 
I love Computer Science and am fascinated by programming. 
Programming is the art of expressing \textit{uncertainty} by \textit{certain} 
means. I'm impressed by Philosophy, especially Philosophy of Mathematics and 
Logic; we tend to see math as a flawless endeavor, while philosophy shows us 
the opposite---vide the quest for certainty that led to the basis of computer 
science. I love learning and doing work that is \textit{meaningful} to me. We are 
not just Turing Machines programmed to perform some action. I also happen to 
enjoy the gracious tastes of \textit{beer}---more or less as in the Reinheitsgebot, 
not the crazy mix of stuff we commonly see.}

\end{document}
