\documentclass[a4paper,10pt]{article}

\usepackage{marvosym}
\usepackage{fontspec}
\usepackage{xunicode,xltxtra,url,parskip}
\RequirePackage{color,graphicx}
\usepackage[usenames,dvipsnames]{xcolor}
\usepackage[big]{layaureo}
% \usepackage{fullpage}
\usepackage{supertabular}
\usepackage{titlesec}
\usepackage{multicol}
\usepackage{multirow}
\usepackage{longtable}
\usepackage{rotating}
\usepackage{ifthen}
\usepackage{hyperref}
\usepackage[absolute]{textpos}
\usepackage{enumitem}
\usepackage[normalem]{ulem}
% \renewcommand{\labelitemi}{\textcolor{lightg}{\symbol{"00BB}}}

% Prefixes bibtex output with "C" (for conference)
\makeatletter
\renewcommand*{\@biblabel}[1]{\hfill[C#1]}
\makeatother

\pagestyle{empty}

%Setup hyperref package, and colours for links
\definecolor{linkcolour}{rgb}{0,0.2,0.6}
\hypersetup{colorlinks,breaklinks,urlcolor=linkcolour, linkcolor=linkcolour}

%Color
\definecolor{lightg}{HTML}{999999}
\definecolor{medg}{HTML}{666666}
\definecolor{darkg}{HTML}{333333}

% Bullets
\definecolor{noteone}{HTML}{999999}
\definecolor{notetwo}{HTML}{848484}
\definecolor{notethree}{HTML}{424242}
\definecolor{notefour}{HTML}{212121}
\definecolor{notefive}{HTML}{000000}

\newcommand{\fivenotes}{%
	\textcolor{noteone}{\symbol{"2022}}
	\textcolor{notetwo}{\symbol{"2022}}
	\textcolor{notethree}{\symbol{"2022}}
	\textcolor{notefour}{\symbol{"2022}}
	\textcolor{notefive}{\symbol{"2022}}
}
\newcommand{\fournotes}{%
	\textcolor{noteone}{\symbol{"2022}}
	\textcolor{notetwo}{\symbol{"2022}}
	\textcolor{notethree}{\symbol{"2022}}
	\textcolor{notefour}{\symbol{"2022}}
	\textcolor{white}{\symbol{"2022}}
}
\newcommand{\threenotes}{%
	\textcolor{noteone}{\symbol{"2022}}
	\textcolor{notetwo}{\symbol{"2022}}
	\textcolor{notethree}{\symbol{"2022}}
	\textcolor{white}{\symbol{"2022}}
	\textcolor{white}{\symbol{"2022}}
}
\newcommand{\twonotes}{%
	\textcolor{noteone}{\symbol{"2022}}
	\textcolor{notetwo}{\symbol{"2022}}
	\textcolor{white}{\symbol{"2022}}
	\textcolor{white}{\symbol{"2022}}
	\textcolor{white}{\symbol{"2022}}
}
\newcommand{\onenote}{%
	\textcolor{noteone}{\symbol{"2022}}
	\textcolor{white}{\symbol{"2022}}
	\textcolor{white}{\symbol{"2022}}
	\textcolor{white}{\symbol{"2022}}
	\textcolor{white}{\symbol{"2022}}
}

%FONTS
\defaultfontfeatures{Mapping=tex-text}
\setmainfont[SmallCapsFont = Fontin SmallCaps]{Fontin}

\font\lighttext=''Baskerville-Normal:color=787878'' at 10pt
\font\lighttextweb=''Baskerville-Normal:color=FF1493'' at 10pt

%CV Sections inspired by: 
%http://stefano.italians.nl/archives/26
\titleformat{\section}{\Large\scshape\raggedright}{}{0em}{}[\titlerule]
\titlespacing{\section}{0pt}{3pt}{3pt}
%Tweak a bit the top margin

%\addtolength{\voffset}{-1.3cm}

%-------------WATERMARK TEST [**not part of a CV**]---------------
\TPGrid[30mm,30mm]{30}{60}
%\setlength{\TPHorizModule}{30mm}
%\setlength{\TPVertModule}{\TPHorizModule}
%\textblockorigin{2mm}{0.65\paperheight}
\setlength{\parindent}{0pt}


\begin{document}

\font\fb=''[cmr10]''

\par{\centering {\Huge\uline{{Guilherme Salazar}}}\bigskip\par\vspace{8ex}}

\section{Personal Data}

\begin{tabular}{r p{3.5in}}
  \textsc{Birth:} & 13 September 1991 \hspace{0.5em}---\hspace{0.5em} Goiânia, Goiás, Brazil\\
  % \textsc{Phone:}     & 803-269-5858\\
  \textsc{email:}     & \href{mailto:gmesalazar@gmail.com}{gmesalazar@gmail.com} | \href{mailto:gmesalazar@acm.org}{gmesalazar@acm.org}\\
  \textsc{Social:} & \href{https://github.com/gmesalazar/}{https://github.com/gmesalazar/}
\end{tabular}

\section{Education}
\begin{tabular}{r|p{11cm}}

  \textsc{2010---2011} & (changed major) Bachelor's in \textsc{Software Engineering}\\ &
  \emph{Instituto de Informática}\\ &
  \textbf{Universidade Federal de Goiás, Brazil}
  \\\multicolumn{2}{c}{} \\

  \textsc{2011---2015} & Bachelor's in \textsc{Computer Science}\\ &
  \emph{Instituto de Informática}\\ &
  \textbf{Universidade Federal de Goiás, Brazil}
  \\\multicolumn{2}{c}{} \\

  \textsc{2012---2013} & Non-Degree Program\\ &
  \emph{College of Engineering \& Computing}\\ &
  \textbf{University of South Carolina, United States}
  \\\multicolumn{2}{c}{} \\

\end{tabular}

\section{Supplemental Courses}
\begin{tabular}{r|p{11cm}}
  \textsc{2010---2010} & Mini course on Aggregation of Software Engineering Experiments\\ &
  \emph{Instituto de Informática} \\ &
  \emph{Universidade Federal de Goiás}\\ &
  Hour load: 8 hours
  \\\multicolumn{2}{c}{} \\

  \textsc{2010---2010} & Mini course on Introduction to Metrics and Measurement of Software\\ &
  \emph{Instituto de Informática} \\ &
  \emph{Universidade Federal de Goiás}\\ &
  Hour load: 8 hours
  \\\multicolumn{2}{c}{} \\

  \textsc{2010---2010} & Mini course on Parallel Programming in CUDA\\ &
  \emph{Instituto de Informática} \\ &
  \emph{Universidade Federal de Goiás}\\ &
  Hour load: 10 hours
  \\\multicolumn{2}{c}{} \\

  \textsc{2010---2011} & English Language Course\\ &
  \emph{Cultural Norte-Americano} \\ &
  Hour load: 120 hours
  \\\multicolumn{2}{c}{} \\

  \textsc{2012---2012} & Summer Pre-Academic Program\\ &
  \emph{College of Health \& Human Services}\\ &
  \emph{University of North Carolina}\\ &
  Hour load: 115 hours
  \\\multicolumn{2}{c}{} \\
\end{tabular}

\pagebreak

\section{Work Experience}
\begin{tabular}{r|p{11cm}}

  \textsc{2011---2011}
  & Teaching Assistant --- \emph{Software Construction}\\
  &\emph{Instituto de Informática, Universidade Federal de Goiás}\\
  &\emph{Dr. Fabbrizio Soares}
  \\\multicolumn{2}{c}{} \\

  \textsc{2013---2013}
  & Software Developer\\
  &\emph{College of Engineering \& Computing, University of South Carolina}\\
  &\footnotesize{Development of an Electronic Health Records application in the Google App Engine platform.}\\
  &\emph{Dr. \href{http://jmvidal.cse.sc.edu}{José M. Vidal}}
  \\\multicolumn{2}{c}{} \\

  \textsc{2014---2014}
  & Teaching Assistant --- \emph{Operating Systems}\\
  &\emph{Instituto de Informática, Universidade Federal de Goiás}\\
  &\emph{Dr. \href{http://inf.ufg.br/~ricardo/}{Ricardo Rocha}}
  \\\multicolumn{2}{c}{} \\

  \textsc{2015}
  & Teaching Assistant --- \emph{Formal Languages and Automata}\\
  &\emph{Instituto de Informática, Universidade Federal de Goiás}\\
  &\emph{Dra. {Márcia Cappelle}}
  \\\multicolumn{2}{c}{} \\


\end{tabular}

%\setlength{\parindent}{1cm}

\section{Research Experience}
\begin{tabular}{r|p{11cm}}

  \textsc{2011---2011}
  & Research Assistant \\
  &\emph{Universidade Federal de Goiás}\\
  &\emph{Supporting the Development of Concurrent/Distributed Systems}\\
  &\indent \footnotesize{Study of alternative models and mechanisms of coordination and communication in event-driven
    concurrent/distributed systems, as well as investigation of the contribution of dynamic programming 
    languages (e.g., Lua) in the development of higher-level concurrency abstractions.}\\
  &\emph{Dr. \href{http://inf.ufg.br/~brunoos/}{Bruno O. Silvestre}}
  \\\multicolumn{2}{c}{} \\

  \textsc{2011---2012}
  & Research Assistant \\
  &\emph{Rede Nacional de Ensino e Pesquisa}\\
  &\emph{Building Smart Cities: From the Instrumentation of Environments to the Development of Applications}\\
  &\footnotesize{Joint effort of 18 Brazilian universities, this project aims at building an infrastructure of 
    implementation, computation, and communication for the viabilization of smart cities. In particular, my main
    task was to study, experiment, and evaluate implementations of the IP stack for low power devices (i.e., 
    wireless sensor network nodes).}\\
  &\emph{Dr. \href{http://inf.ufg.br/~brunoos/}{Bruno O. Silvestre}}
  \\\multicolumn{2}{c}{} \\

\end{tabular}

\nocite{salazar:1}

\renewcommand\refname{Publications} 
\bibliography{cv}
\bibliographystyle{plainyr-rev}

%% \section{Presentations}
%% \begin{enumerate}
%%   \renewcommand{\labelenumi}{[P\arabic{enumi}] }
%% \item \emph{Proposta de um Modelo para o Processamento de Eventos Concorrentes no ALua}. Encontro Anual de Computação, Universidade Federal de Goiás. October 26, 2011.\\
%% \end{enumerate}

\section{Academic Societies}

\begin{tabular}{r|p{11cm}}
  \textsc{2011---Current}
  & Student Member of the Association for Computing Machinery\\
  & \footnotesize{\url{http://member.acm.org/~gmesalazar}}
  \\\multicolumn{2}{c}{}\\

  \textsc{2013---Current}
  & Alpha Lambda Delta Honor Society\\
  & \footnotesize{Iniciated as a member at the University of South Carolina, Columbia, on February 11, 2013}\\
  \multicolumn{2}{c}{}\\
  
\end{tabular}

\section{Honors/Awards}
\begin{tabular}{r|p{11cm}}
  \textsc{2011}
  & Best Paper Award\\
  & Encontro Anual de Computação, Universidade Federal de Goiás.\\
  \multicolumn{2}{c}{}\\

  \textsc{2012}
  & Study Abroad Scholarship Recipient\\
  & Ministry of Education, Government of Brazil\\
  \multicolumn{2}{c}{}\\

  \textsc{2012---2013}
  & Dean's Honor List\\
  & College of Engineering \& Computing, University of South Carolina\\
  \multicolumn{2}{c}{}\\
\end{tabular}

\section{Skills \& Languages}
\begin{multicols}{3}
  \begin{itemize}
    \renewcommand{\labelitemi}{\textcolor{lightg}{\symbol{"00BB}}}
    \setlength{\itemsep}{1pt}
    \setlength{\parskip}{0pt}
    \setlength{\parsep}{0pt}
  \item C \hfill \threenotes 
  \item Java \hfill \threenotes
  \item Lua \hfill \twonotes
  \item nesC \hfill \twonotes
  \item Python \hfill \twonotes
  \item JavaScript \hfill \twonotes
  \item HTML/CSS \hfill \threenotes
  \item SQL \hfill \twonotes
  \item Android Dev \hfill \twonotes
  \item App Engine \hfill \twonotes
  \item GNU/Linux \hfill \threenotes
  \item Windows \hfill \threenotes
  \item Portuguese \hfill \fournotes
  \item English \hfill \threenotes
  \end{itemize} 
\end{multicols}

\vspace{1em}

\begin{center}
  \parbox{12cm}{
    \onenote Some familiarity, small-scale projects and assignments \\
    \twonotes Implementation-specific experience \\
    \threenotes Quite familiar, used in limited settings as part of larger projects \\
    \fournotes Extensive knowledge
    %    \fivenotes Used in context of large scale, multi-group projects
  }
\end{center}

\end{document}
