\documentclass[a4paper,10pt]{article}

\usepackage{fontspec}
\RequirePackage{color,graphicx}
\usepackage[usenames,dvipsnames]{xcolor}
\usepackage[big]{layaureo}
\usepackage{supertabular}
\usepackage{titlesec}
\usepackage{multicol}
\usepackage{multirow}
\usepackage{longtable}
\usepackage{rotating}
\usepackage{ifthen}
\usepackage{hyperref}
\usepackage[absolute]{textpos}
\usepackage{enumitem}
\usepackage[normalem]{ulem}

\pagestyle{empty}

%Setup hyperref package, and colours for links
\definecolor{linkcolour}{rgb}{0,0.2,0.6}
\hypersetup{colorlinks,breaklinks,urlcolor=linkcolour, linkcolor=linkcolour}

%Color
\definecolor{lightg}{HTML}{999999}
\definecolor{medg}{HTML}{666666}
\definecolor{darkg}{HTML}{333333}

% Bullets
\definecolor{noteone}{HTML}{999999}
\definecolor{notetwo}{HTML}{848484}
\definecolor{notethree}{HTML}{424242}
\definecolor{notefour}{HTML}{212121}
\definecolor{notefive}{HTML}{000000}

\newcommand{\fivenotes}{%
	\textcolor{noteone}{\symbol{"2022}}
	\textcolor{notetwo}{\symbol{"2022}}
	\textcolor{notethree}{\symbol{"2022}}
	\textcolor{notefour}{\symbol{"2022}}
	\textcolor{notefive}{\symbol{"2022}}
}
\newcommand{\fournotes}{%
	\textcolor{noteone}{\symbol{"2022}}
	\textcolor{notetwo}{\symbol{"2022}}
	\textcolor{notethree}{\symbol{"2022}}
	\textcolor{notefour}{\symbol{"2022}}
	\textcolor{white}{\symbol{"2022}}
}
\newcommand{\threenotes}{%
	\textcolor{noteone}{\symbol{"2022}}
	\textcolor{notetwo}{\symbol{"2022}}
	\textcolor{notethree}{\symbol{"2022}}
	\textcolor{white}{\symbol{"2022}}
	\textcolor{white}{\symbol{"2022}}
}
\newcommand{\twonotes}{%
	\textcolor{noteone}{\symbol{"2022}}
	\textcolor{notetwo}{\symbol{"2022}}
	\textcolor{white}{\symbol{"2022}}
	\textcolor{white}{\symbol{"2022}}
	\textcolor{white}{\symbol{"2022}}
}
\newcommand{\onenote}{%
	\textcolor{noteone}{\symbol{"2022}}
	\textcolor{white}{\symbol{"2022}}
	\textcolor{white}{\symbol{"2022}}
	\textcolor{white}{\symbol{"2022}}
	\textcolor{white}{\symbol{"2022}}
}

%FONTS
\defaultfontfeatures{Mapping=tex-text}
\setmainfont[SmallCapsFont = Fontin SmallCaps]{Fontin}

\font\lighttext=''Baskerville-Normal:color=787878'' at 10pt
\font\lighttextweb=''Baskerville-Normal:color=FF1493'' at 10pt

%CV Sections inspired by: 
%http://stefano.italians.nl/archives/26
\titleformat{\section}{\Large\scshape\raggedright}{}{0em}{}[\titlerule]
\titlespacing{\section}{0pt}{3pt}{3pt}
%Tweak a bit the top margin

%\addtolength{\voffset}{-1.3cm}

%-------------WATERMARK TEST [**not part of a CV**]---------------
\TPGrid[30mm,30mm]{30}{60}
%\setlength{\TPHorizModule}{30mm}
%\setlength{\TPVertModule}{\TPHorizModule}
%\textblockorigin{2mm}{0.65\paperheight}
\setlength{\parindent}{0pt}


\begin{document}

\font\fb=''[cmr10]''

\par{\center{\huge\uline{{Guilherme Salazar}}}\bigskip\par\vspace{4ex}}

\section{Personal}

\begin{tabular}{l@{\hskip 0.2cm}l}
  Born & \texttt{on} Sep 13, '91 \texttt{in}
                     \href{https://en.wikipedia.org/wiki/Goiania}{Goiânia}, BR \\
  Mail & \texttt{at} \href{mailto:gmesalazar@gmail.com}
                          {gmesalazar@gmail.com} \\
  Code & \texttt{at} \href{https://www.github.com/salazar/}
                          {GitHub.com/salazar}
\end{tabular}

\section{Education}

\begin{tabular}{r|p{11cm}}
  \textsc{Aug '12--Jul '13} & Computer Science, Visiting\\ &
  \emph{College of Engineering \& Computing}\\ &
  \textbf{University of South Carolina, United States}
  \\\multicolumn{2}{c}{} \\

  \textsc{Aug '11--Jul '16} & Computer Science, B.Sc\\ &
  \emph{Instituto de Informática}\\ &
  \textbf{Universidade Federal de Goiás, Brazil}
  \\\multicolumn{2}{c}{} \\

  \textsc{Mar '10--Jul '11} & Software Engineering, Dropped \\ &
  \emph{Instituto de Informática}\\ &
  \textbf{Universidade Federal de Goiás, Brazil}
\end{tabular}

\section{Courses}

\begin{tabular}{r|p{11cm}}
  \textsc{Jul--Aug '12} & Summer Pre-Academic Program\\ &
  \emph{English Language Training Institute}\\ &
  \emph{University of North Carolina}\\ &
  Hour load: 115 hours
  \\\multicolumn{2}{c}{} \\

  \textsc{Jan '10--Jul '11} & English Language Course\\ &
  \emph{Cultural Norte-Americano} \\ &
  Hour load: 120 hours
  \\\multicolumn{2}{c}{} \\

  \textsc{Oct '10} & Introduction to Metrics and Measurement of Software\\ &
  \emph{Instituto de Informática} \\ &
  \emph{Universidade Federal de Goiás}\\ &
  Hour load: 8 hours
  \\\multicolumn{2}{c}{} \\

  \textsc{Oct '10} & Parallel Programming in CUDA\\ &
  \emph{Instituto de Informática} \\ &
  \emph{Universidade Federal de Goiás}\\ &
  Hour load: 10 hours
\end{tabular}

\section{Work}

\begin{longtable}{r|p{11cm}}

  \textsc{Apr--Aug '16} & Programmer \\
  &\emph{Google Summer of Code, LabLua}\\
  &\emph{\footnotesize{\href{https://goo.gl/g8Mb2M}{I/O API for NetBSD kernel Lua}}}\\
  &\footnotesize{Working on NetBSD kernel Lua: the kernel port of the Lua 
  language lacks some of the libraries available in user space Lua, such as
  the standard 'io' module. Such modules are not freestanding; they rely on
  the operating system to provide their functionality. The main goal of this
  project is to develop I/O bindings to kernel Lua, supporting regular files
  and sockets. More on the \href{https://goo.gl/g8Mb2M}{GSoC platform} and
  \href{https://GitHub.com/salazar/luaio}{GitHub}}
  \\\multicolumn{2}{c}{} \\

  \textsc{Dec '15--Cur}
  & Programmer \\
  &\emph{The NetBSD Foundation}\\
  &\emph{\footnotesize{My sponsored work area is kernel Lua, the Lua language
  port to kernel space, which allows for kernel scripting; in general, my main
  area of interest are the I/O subsystem, especially the VFS layer and networking}}
  \\\multicolumn{2}{c}{} \\
  
  \textsc{Apr--Aug '15}
  & Programmer \\
  &\emph{Google Summer of Code, LabLua}\\
  &\emph{\footnotesize{\href{https://goo.gl/xSl1bW}{Port Lua test
         suite to the NetBSD kernel}}}\\
  &\footnotesize{The Lua interpreter was ported to the NetBSD kernel in a GSoC 
   2010 project, making it possible to interact with kernel subsystems using
   Lua. The main goal of this project was to port Lua test suite to the NetBSD
   kernel; test scripts had to be adapted to remove dependencies from floating-
   point numbers and parts of Lua's standard libraries had to be reimplemented 
   using kernel interfaces. More on \href{https://goo.gl/xSl1bW}{Google Melange}
   and \href{https://GitHub.com/salazar/luatests}{GitHub}}
  \\\multicolumn{2}{c}{} \\

  \textsc{Jan--Jul '13}
  & Programmer \\
  &\emph{College of Engineering \& Computing, University of South Carolina}\\
  &\footnotesize{Development of a GAE app for USC's Medical School}\\
  &\emph{Under Dr. \href{http://jmvidal.cse.sc.edu}{José Vidal}}
  \\\multicolumn{2}{c}{} \\

  \textsc{Sep '11--Jul '12}
  & Research Assistant \\
  &\emph{Rede Nacional de Ensino e Pesquisa}\\
  &\footnotesize{Study, experimentation, and Evaluation of IP stack implementations 
  for low-power devices (e.g., TelosB and MicaZ motes). See 
            \href{http://www.nr2.ufpr.br/cia2/}{the page} of the main project}\\
  &\emph{Under Dr. \href{http://inf.ufg.br/~brunoos/}{Bruno Silvestre}}
  \\\multicolumn{2}{c}{} \\

  \textsc{Mar--Sep '11}
  & Research Assistant \\
  &\emph{Instituto de Informática, Universidade Federal de Goiás}\\
  &\emph{Supporting the Development of Concurrent and Distributed Systems}\\
  &\indent \footnotesize{Study of models and mechanisms of coordination and 
        communication in event-driven concurrent and distributed systems; the 
        use of Lua in the development of higher-level concurrency abstractions.}\\
  &\emph{Under Dr. \href{http://inf.ufg.br/~brunoos/}{Bruno Silvestre}}
  \\\multicolumn{2}{c}{} \\

  \textsc{'11, '14, '15}
  & Teaching Assistant \\
  &\emph{Instituto de Informática, Universidade Federal de Goiás} \\
  &\emph{Mar--Jul '14: \footnotesize{Operating Systems,
                       under Dr. Ricardo Rocha}} \\
  &\emph{Mar--Jul '15: \footnotesize{Formal Languages and Automata Theory,
                       under Dr. Marcia Capelle}} \\
  &\emph{Aug--Dec '11: \footnotesize{Software Construction,
                       under Dr. Fabrizzio Soares}}
\end{longtable}

\section{Publications}
\begin{enumerate}
  \renewcommand{\labelenumi}{[\arabic{enumi}] }
  \item Bruno Silvestre, Guilherme Salazar. \emph{Proposta de um Modelo para o
        Processamento de Eventos Concorrentes no ALua.} Encontro Anual de
        Computação, Universidade Federal de Goiás. October 26, 2011
\end{enumerate}

\section{Associations}

\begin{tabular}{r|p{11cm}}

  \textsc{May '16--Cur}
  & Brazilian Computing Society\\
  \multicolumn{2}{c}{}\\

  \textsc{Dec '15--Cur}
  & The NetBSD Foundation
  \\\multicolumn{2}{c}{}\\

  \textsc{Feb '13--Cur}
  & Alpha Lambda Delta Honor Society\\
  \multicolumn{2}{c}{}\\

  \textsc{Dec '11--Cur}
  & Association for Computing Machinery\\

\end{tabular}

\section{Awards}
\begin{tabular}{r|p{11cm}}

  \textsc{2012--2013}
  & Dean's Honor List\\
  & College of Engineering \& Computing, University of South Carolina\\
  \multicolumn{2}{c}{}\\

  \textsc{2012}
  & Study Abroad Scholarship\\
  & Ministry of Education, Government of Brazil\\
  \multicolumn{2}{c}{}\\

  \textsc{2011}
  & Best Paper Award\\
  & Encontro Anual de Computação, Universidade Federal de Goiás\\

\end{tabular}

\section{Skills}

\begin{multicols}{3}
  \begin{itemize}
    \renewcommand{\labelitemi}{\textcolor{lightg}{\symbol{"00BB}}}
    \setlength{\itemsep}{1pt}
    \setlength{\parskip}{0pt}
    \setlength{\parsep}{0pt}
  \item C \hfill \threenotes 
  \item Java \hfill \twonotes
  \item Lua \hfill \threenotes
  \item nesC \hfill \onenote
  \item Python \hfill \twonotes
  \item JavaScript \hfill \twonotes
  \item HTML/CSS \hfill \twonotes
  \item \LaTeX \hfill \onenote
  \item SQL \hfill \twonotes
  \item Android \hfill \twonotes
  \item GAE \hfill \twonotes
  \item GNU/Linux \hfill \threenotes
  \item NetBSD \hfill \twonotes
  \item Windows \hfill \onenote
  \item Vagrant \hfill \twonotes
  \item Git \hfill \twonotes
  \item GitHub \hfill \twonotes
  \item SVN \hfill \onenote
  \item CVS \hfill \onenote
  \item BR Portuguese \hfill \twonotes
  \item US English \hfill \twonotes
  \end{itemize} 
\end{multicols}

\vspace{1em}

\begin{center}
\parbox[c]{8cm}{
  \onenote Familiar; small-scale projects \\
  \twonotes Quite familiar; used in larger projects \\
  \threenotes Extensive knowledge
}
\end{center}

\end{document}
