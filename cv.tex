\documentclass[a4paper,10pt]{article}

\usepackage{marvosym}
\usepackage{fontspec}
\usepackage{xunicode,xltxtra,url,parskip}
\RequirePackage{color,graphicx}
\usepackage[usenames,dvipsnames]{xcolor}
\usepackage[big]{layaureo}
% \usepackage{fullpage}
\usepackage{supertabular}
\usepackage{titlesec}
\usepackage{multicol}
\usepackage{multirow}
\usepackage{longtable}
\usepackage{rotating}
\usepackage{ifthen}
\usepackage{hyperref}
\usepackage[absolute]{textpos}
\usepackage{enumitem}
\usepackage[normalem]{ulem}
% \renewcommand{\labelitemi}{\textcolor{lightg}{\symbol{"00BB}}}

% Prefixes bibtex output with "C" (for conference)
\makeatletter
\renewcommand*{\@biblabel}[1]{\hfill[C#1]}
\makeatother

\pagestyle{empty}

%Setup hyperref package, and colours for links
\definecolor{linkcolour}{rgb}{0,0.2,0.6}
\hypersetup{colorlinks,breaklinks,urlcolor=linkcolour, linkcolor=linkcolour}

%Color
\definecolor{lightg}{HTML}{999999}
\definecolor{medg}{HTML}{666666}
\definecolor{darkg}{HTML}{333333}

% Bullets
\definecolor{noteone}{HTML}{999999}
\definecolor{notetwo}{HTML}{848484}
\definecolor{notethree}{HTML}{424242}
\definecolor{notefour}{HTML}{212121}
\definecolor{notefive}{HTML}{000000}

\newcommand{\fivenotes}{%
	\textcolor{noteone}{\symbol{"2022}}
	\textcolor{notetwo}{\symbol{"2022}}
	\textcolor{notethree}{\symbol{"2022}}
	\textcolor{notefour}{\symbol{"2022}}
	\textcolor{notefive}{\symbol{"2022}}
}
\newcommand{\fournotes}{%
	\textcolor{noteone}{\symbol{"2022}}
	\textcolor{notetwo}{\symbol{"2022}}
	\textcolor{notethree}{\symbol{"2022}}
	\textcolor{notefour}{\symbol{"2022}}
	\textcolor{white}{\symbol{"2022}}
}
\newcommand{\threenotes}{%
	\textcolor{noteone}{\symbol{"2022}}
	\textcolor{notetwo}{\symbol{"2022}}
	\textcolor{notethree}{\symbol{"2022}}
	\textcolor{white}{\symbol{"2022}}
	\textcolor{white}{\symbol{"2022}}
}
\newcommand{\twonotes}{%
	\textcolor{noteone}{\symbol{"2022}}
	\textcolor{notetwo}{\symbol{"2022}}
	\textcolor{white}{\symbol{"2022}}
	\textcolor{white}{\symbol{"2022}}
	\textcolor{white}{\symbol{"2022}}
}
\newcommand{\onenote}{%
	\textcolor{noteone}{\symbol{"2022}}
	\textcolor{white}{\symbol{"2022}}
	\textcolor{white}{\symbol{"2022}}
	\textcolor{white}{\symbol{"2022}}
	\textcolor{white}{\symbol{"2022}}
}

%FONTS
\defaultfontfeatures{Mapping=tex-text}
\setmainfont[SmallCapsFont = Fontin SmallCaps]{Fontin}

\font\lighttext=''Baskerville-Normal:color=787878'' at 10pt
\font\lighttextweb=''Baskerville-Normal:color=FF1493'' at 10pt

%CV Sections inspired by: 
%http://stefano.italians.nl/archives/26
\titleformat{\section}{\Large\scshape\raggedright}{}{0em}{}[\titlerule]
\titlespacing{\section}{0pt}{3pt}{3pt}
%Tweak a bit the top margin

%\addtolength{\voffset}{-1.3cm}

%-------------WATERMARK TEST [**not part of a CV**]---------------
\TPGrid[30mm,30mm]{30}{60}
%\setlength{\TPHorizModule}{30mm}
%\setlength{\TPVertModule}{\TPHorizModule}
%\textblockorigin{2mm}{0.65\paperheight}
\setlength{\parindent}{0pt}


\begin{document}

\font\fb=''[cmr10]''

\par{\centering {\Huge\uline{{Guilherme Salazar}}}\bigskip\par\vspace{8ex}}

\section{Personal}

\begin{tabular}{r p{3.5in}}
  \textsc{Birth:} & 13 September 1991 \hspace{0.5em}---\hspace{0.5em} Goiânia, Goiás, Brazil\\
  % \textsc{Phone:}     & 803-269-5858\\
  \textsc{email:}     & \href{mailto:gmesalazar@gmail.com}{gmesalazar@gmail.com} \\
  \textsc{GitHub:} & \href{https://www.github.com/gmesalazar/}
  					{https://www.github.com/gmesalazar}\\
\end{tabular}

\section{Education}
\begin{tabular}{r|p{11cm}}

  \textsc{Aug $'$12---Jul $'$13} & Computer Science, Visiting\\ &
  \emph{College of Engineering \& Computing}\\ &
  \textbf{University of South Carolina, United States}
  \\\multicolumn{2}{c}{} \\

  \textsc{Aug $'$11---Jul $'$16} & Computer Science, B.Sc\\ &
  \emph{Instituto de Informática}\\ &
  \textbf{Universidade Federal de Goiás, Brazil}
  \\\multicolumn{2}{c}{} \\

  \textsc{Mar $'$10---Aug $'$11} & Software Engineering, Got too bored and moved to CS :)\\ &
  \emph{Instituto de Informática}\\ &
  \textbf{Universidade Federal de Goiás, Brazil}
  \\\multicolumn{2}{c}{} \\

\end{tabular}

\section{Courses}
\begin{tabular}{r|p{11cm}}

  \textsc{Jul---Aug $'$12} & Summer Pre-Academic Program\\ &
  \emph{College of Health \& Human Services}\\ &
  \emph{University of North Carolina}\\ &
  Hour load: 115 hours
  \\\multicolumn{2}{c}{} \\

  \textsc{Jan $'$10--- Jul $'$11} & English Language Course\\ &
  \emph{Cultural Norte-Americano} \\ &
  Hour load: 120 hours
  \\\multicolumn{2}{c}{} \\

  \textsc{Oct $'$10} & Aggregation of Software Engineering Experiments\\ &
  \emph{Instituto de Informática} \\ &
  \emph{Universidade Federal de Goiás}\\ &
  Hour load: 8 hours
  \\\multicolumn{2}{c}{} \\

  \textsc{Oct $'$10} & Introduction to Metrics and Measurement of Software\\ &
  \emph{Instituto de Informática} \\ &
  \emph{Universidade Federal de Goiás}\\ &
  Hour load: 8 hours
  \\\multicolumn{2}{c}{} \\

  \textsc{Oct $'$10} & Parallel Programming in CUDA\\ &
  \emph{Instituto de Informática} \\ &
  \emph{Universidade Federal de Goiás}\\ &
  Hour load: 10 hours
  \\\multicolumn{2}{c}{} \\

\end{tabular}

\pagebreak

\section{Work Experience}
\begin{tabular}{r|p{11cm}}

  \textsc{Dec $'$15---Cur}
  & Volunteer Developer\\
  &\emph{The NetBSD Foundation}\\
  &\emph{\footnotesize{Works with kernel Lua, the port of the Lua language to
  						kernel space; currently, aims to bring user level
						Lua standard liraries to kernel space}}
  \\\multicolumn{2}{c}{} \\
  
  \textsc{May---Aug $'$15}
  & Software Developer\\
  &\emph{Google Summer of Code, Google Inc., LabLua}\\
  &\emph{\footnotesize{\href{https://www.google-melange.com/gsoc/project/details/google/gsoc2015/gmesalazar/5741031244955648}{Port Lua test suite to the NetBSD kernel}}}\\
  &\emph{Under Lourival Vieira Neto}
  \\\multicolumn{2}{c}{} \\

  \textsc{Mar---Jul $'$15}
  & Teaching Assistant --- \emph{Formal Languages and Automata}\\
  &\emph{Instituto de Informática, Universidade Federal de Goiás}\\
  &\emph{Under Dr. {Márcia Cappelle}}
  \\\multicolumn{2}{c}{} \\

  \textsc{Mar---Jul $'$14}
  & Teaching Assistant --- \emph{Operating Systems}\\
  &\emph{Instituto de Informática, Universidade Federal de Goiás}\\
  &\emph{Under Dr. \href{http://inf.ufg.br/~ricardo/}{Ricardo Rocha}}
  \\\multicolumn{2}{c}{} \\

  \textsc{Jan---Jul $'$13}
  & Software Developer\\
  &\emph{College of Engineering \& Computing, University of South Carolina}\\
  &\footnotesize{Development of a GAE app for USC's Medical School}\\
  &\emph{Under Dr. \href{http://jmvidal.cse.sc.edu}{José M. Vidal}}
  \\\multicolumn{2}{c}{} \\

  \textsc{Aug---Dec $'$11}
  & Teaching Assistant --- \emph{Software Construction}\\
  &\emph{Instituto de Informática, Universidade Federal de Goiás}\\
  &\emph{Under Dr. Fabbrizio Soares}
  \\\multicolumn{2}{c}{} \\

\end{tabular}

%\setlength{\parindent}{1cm}

\section{Research Experience}
\begin{tabular}{r|p{11cm}}

  \textsc{Sep $'$11---Jul $'$12}
  & Research Assistant \\
  &\emph{Rede Nacional de Ensino e Pesquisa}\\
  &\emph{Building Smart Cities: From the Instrumentation of Environments to the Development of Applications}\\
  &\footnotesize{Joint effort of 18 Brazilian universities, this project aims at building an infrastructure of 
    implementation, computation, and communication for the viabilization of smart cities. In particular, my main
    task was to study, experiment, and evaluate implementations of the IP stack for low power devices (i.e., 
    wireless sensor network nodes).}\\
  &\emph{Dr. \href{http://inf.ufg.br/~brunoos/}{Bruno O. Silvestre}}
  \\\multicolumn{2}{c}{} \\

  \textsc{Mar---Sep $'$11}
  & Research Assistant \\
  &\emph{Universidade Federal de Goiás}\\
  &\emph{Supporting the Development of Concurrent/Distributed Systems}\\
  &\indent \footnotesize{Study of alternative models and mechanisms of coordination and communication in event-driven
    concurrent/distributed systems; investigation of the contribution of dynamic programming 
    languages (e.g., Lua) in the development of higher-level concurrency abstractions.}\\
  &\emph{Dr. \href{http://inf.ufg.br/~brunoos/}{Bruno O. Silvestre}}
  \\\multicolumn{2}{c}{} \\

\end{tabular}

\nocite{salazar:1}

\renewcommand\refname{Publications} 
\bibliography{cv}
\bibliographystyle{plainyr-rev}

%% \section{Presentations}
%% \begin{enumerate}
%%   \renewcommand{\labelenumi}{[P\arabic{enumi}] }
%% \item \emph{Proposta de um Modelo para o Processamento de Eventos Concorrentes no ALua}. Encontro Anual de Computação, Universidade Federal de Goiás. October 26, 2011.\\
%% \end{enumerate}

\section{Associations}

\begin{tabular}{r|p{11cm}}

  \textsc{Dec $'$15---Cur}
  & The NetBSD Foundation\\
  & \footnotesize{\url{http://www.netbsd.org/~salazar}}
  \\\multicolumn{2}{c}{}\\

  \textsc{Feb $'$13---Cur}
  & Alpha Lambda Delta Honor Society\\
  & \footnotesize{Iniciated at the University of South Carolina, Columbia, on February 11, 2013}\\
  \multicolumn{2}{c}{}\\

  \textsc{Dec $'$11---Cur}
  & Association for Computing Machinery\\
  & \footnotesize{\url{http://member.acm.org/~gmesalazar}}
  \\\multicolumn{2}{c}{}\\
  
\end{tabular}

\section{Awards}
\begin{tabular}{r|p{11cm}}

  \textsc{$'$12---$'$13}
  & Dean's Honor List\\
  & College of Engineering \& Computing, University of South Carolina\\
  \multicolumn{2}{c}{}\\

  \textsc{$'$12}
  & Study Abroad Scholarship\\
  & Ministry of Education, Government of Brazil\\
  \multicolumn{2}{c}{}\\

  \textsc{$'$11}
  & Best Paper Award\\
  & Encontro Anual de Computação, Universidade Federal de Goiás\\
  \multicolumn{2}{c}{}\\

\end{tabular}

\section{Tools}
\centering{Some tools I've used or regularly use:}

\begin{multicols}{3}
  \begin{itemize}
    \renewcommand{\labelitemi}{\textcolor{lightg}{\symbol{"00BB}}}
    \setlength{\itemsep}{1pt}
    \setlength{\parskip}{0pt}
    \setlength{\parsep}{0pt}
  \item C \hfill \threenotes 
  \item Java \hfill \threenotes
  \item Lua \hfill \threenotes
  \item nesC \hfill \onenote
  \item Python \hfill \twonotes
  \item JavaScript \hfill \twonotes
  \item HTML/CSS \hfill \twonotes
  \item \LaTeX \hfill \twonotes
  \item SQL \hfill \twonotes
  \item Android \hfill \threenotes
  \item GAE \hfill \twonotes
  \item GNU/Linux \hfill \threenotes
  \item NetBSD \hfill \twonotes
  \item Windows \hfill \twonotes
  \item Portuguese \hfill \fournotes
  \item English \hfill \threenotes
  \item Vagrant \hfill \threenotes
  \item Git \hfill \twonotes
  \item GitHub \hfill \twonotes
  \item SVN \hfill \onenote
  \item CVS \hfill \onenote
  \end{itemize} 
\end{multicols}

\vspace{1em}

\begin{center}
  \parbox{12cm}{
    \onenote Some familiarity, small-scale projects and assignments \\
    \twonotes Implementation-specific experience \\
    \threenotes Quite familiar, used in limited settings as part of larger projects \\
    \fournotes Extensive knowledge
    %    \fivenotes Used in context of large scale, multi-group projects
  }
\end{center}

\section{Misc}
\centering{A lot of stuff interest me. Open-source software is one of them. I'm by no
means a fanatic, but I believe openness---not only in software---can lead to a 
better world. I love Computer Science and am fascinated by programming. 
Programming is the art of expressing \textit{uncertainty} by \textit{certain} 
means. I'm impressed by Philosophy, especially Philosophy of Mathematics and 
Logic; we tend to see math as a flawless endeavor, while philosophy shows us 
the opposite---vide the quest for certainty that led to the basis of computer 
science. I love learning and doing \textit{meaninful} work. Oh, I also happen to
enjoy the gracious tastes of hops with malted barley.}

\end{document}
