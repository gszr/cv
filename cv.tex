\documentclass[a4paper,10pt]{article}

\usepackage{fontspec}
\RequirePackage{color,graphicx}
\usepackage[usenames,dvipsnames]{xcolor}
\usepackage[big]{layaureo}
\usepackage{supertabular}
\usepackage{titlesec}
\usepackage{multicol}
\usepackage{multirow}
\usepackage{longtable}
\usepackage{rotating}
\usepackage{ifthen}
\usepackage{hyperref}
\usepackage[absolute]{textpos}
\usepackage{enumitem}
\usepackage[normalem]{ulem}

\pagestyle{empty}

%Setup hyperref package, and colours for links
\definecolor{linkcolour}{rgb}{0,0.2,0.6}
\hypersetup{colorlinks,breaklinks,urlcolor=linkcolour, linkcolor=linkcolour}

%Color
\definecolor{lightg}{HTML}{999999}
\definecolor{medg}{HTML}{666666}
\definecolor{darkg}{HTML}{333333}

% Bullets
\definecolor{noteone}{HTML}{999999}
\definecolor{notetwo}{HTML}{848484}
\definecolor{notethree}{HTML}{424242}
\definecolor{notefour}{HTML}{212121}
\definecolor{notefive}{HTML}{000000}

\newcommand{\fivenotes}{%
	\textcolor{noteone}{\symbol{"2022}}
	\textcolor{notetwo}{\symbol{"2022}}
	\textcolor{notethree}{\symbol{"2022}}
	\textcolor{notefour}{\symbol{"2022}}
	\textcolor{notefive}{\symbol{"2022}}
}
\newcommand{\fournotes}{%
	\textcolor{noteone}{\symbol{"2022}}
	\textcolor{notetwo}{\symbol{"2022}}
	\textcolor{notethree}{\symbol{"2022}}
	\textcolor{notefour}{\symbol{"2022}}
	\textcolor{white}{\symbol{"2022}}
}
\newcommand{\threenotes}{%
	\textcolor{noteone}{\symbol{"2022}}
	\textcolor{notetwo}{\symbol{"2022}}
	\textcolor{notethree}{\symbol{"2022}}
	\textcolor{white}{\symbol{"2022}}
	\textcolor{white}{\symbol{"2022}}
}
\newcommand{\twonotes}{%
	\textcolor{noteone}{\symbol{"2022}}
	\textcolor{notetwo}{\symbol{"2022}}
	\textcolor{white}{\symbol{"2022}}
	\textcolor{white}{\symbol{"2022}}
	\textcolor{white}{\symbol{"2022}}
}
\newcommand{\onenote}{%
	\textcolor{noteone}{\symbol{"2022}}
	\textcolor{white}{\symbol{"2022}}
	\textcolor{white}{\symbol{"2022}}
	\textcolor{white}{\symbol{"2022}}
	\textcolor{white}{\symbol{"2022}}
}

%FONTS
\defaultfontfeatures{Mapping=tex-text}
\setmainfont[SmallCapsFont = Fontin SmallCaps]{Fontin}

\font\lighttext=''Baskerville-Normal:color=787878'' at 10pt
\font\lighttextweb=''Baskerville-Normal:color=FF1493'' at 10pt

%CV Sections inspired by: 
%http://stefano.italians.nl/archives/26
\titleformat{\section}{\Large\scshape\raggedright}{}{0em}{}[\titlerule]
\titlespacing{\section}{0pt}{3pt}{3pt}
%Tweak a bit the top margin

%\addtolength{\voffset}{-1.3cm}

%-------------WATERMARK TEST [**not part of a CV**]---------------
\TPGrid[30mm,30mm]{30}{60}
%\setlength{\TPHorizModule}{30mm}
%\setlength{\TPVertModule}{\TPHorizModule}
%\textblockorigin{2mm}{0.65\paperheight}
\setlength{\parindent}{0pt}


\begin{document}

\font\fb=''[cmr10]''

\center{\huge\uline{{Guilherme Salazar}}}
\center{\large{Born on Sep 13, 1991, in Goiânia, BR} \\
\href{mailto:gmesalazar@gmail.com}{gmesalazar@gmail.com} \\
\href{https://www.github.com/salazar/}{github.com/salazar}\vspace{6ex}
}

\section{Education}

\begin{tabular}{r|p{11cm}}
  \textsc{Aug '12--Jul '13} & Computer Science, Visiting\\ &
  \emph{College of Engineering \& Computing}\\ &
  \emph{University of South Carolina, United States}
  \\\multicolumn{2}{c}{} \\

  \textsc{Mar '10--Jul '16} & Computer Science, B.Sc\\ &
  Software Engineering, B.Sc, dropped \\ &
  \emph{Instituto de Informática}\\ &
  \emph{Universidade Federal de Goiás, Brazil} \\
\end{tabular}

\section{Courses}

\begin{tabular}{r|p{11cm}}
  \textsc{Jul--Aug '12} & Summer Pre-Academic Program\\ &
  \emph{English Language Training Institute}\\ &
  \emph{University of North Carolina}\\ &
  Hour load: 115 hours
  \\\multicolumn{2}{c}{} \\

  \textsc{Jan '10--Jul '11} & English Language Course\\ &
  \emph{Cultural Norte-Americano} \\ &
  Hour load: 120 hours
\end{tabular}

\section{Work}

\begin{longtable}{r|p{11cm}}

  \textsc{Apr--Aug '16} & Student Developer \\
  &\emph{Google Summer of Code, LabLua}\\
  &\emph{\footnotesize{I/O API for NetBSD kernel Lua}}\\
  &\footnotesize{The kernel port of the Lua language lacks some of the libraries
  available in user space Lua, such as the standard 'io', 'os', and 'math' modules. 
  Such modules are not freestanding; they rely on the operating system to provide
  their functionality. In this project I ported Lua 'io' module and developed a 
  sockets library to kernel Lua. See the project on 
  \href{https://GitHub.com/salazar/luaio}{GitHub}}
  \\\multicolumn{2}{c}{} \\

  \textsc{Dec '15--Cur}
  & Developer (Volunteer)\\
  &\emph{The NetBSD Foundation}\\
  &\footnotesize{Maintainer for Lua user and kernel space implementations. I
  also maintain, out of tree, the test suite for kernel space Lua. I've done
  work porting parts of Lua standard libraries to kernel space--e.g., \emph{base, 
  math, io, os} and developing I/O bindings to file systems and sockets. More 
  on \href{https://GitHub.com/salazar/netbsd}{GitHub}}
  \\\multicolumn{2}{c}{} \\
  
  \textsc{Apr--Aug '15}
  & Student Developer \\
  &\emph{Google Summer of Code, LabLua}\\
  &\emph{\footnotesize{Port Lua test
         suite to the NetBSD kernel}}\\
  &\footnotesize{Ported Lua test suite to NetBSD kernel Lua, the port of the Lua
  language to kernel space; test scripts
   had to be adapted to remove dependencies from floating-point numbers and parts
   of Lua's standard libraries had to be reimplemented using kernel interfaces.
   See the project on 
   \href{https://GitHub.com/salazar/luatests}{GitHub}}
  \\\multicolumn{2}{c}{} \\

  \textsc{Jan '14--Jul '15}
  & Programmer \\
  &\emph{Macro Softwares, LTDA}\\
  &\footnotesize{Developed an Android app for package delivery management. The
  app displays the info of each package, such as recipient and address.
  The courier can attach observations in text, audio, take pictures,
  change the status of a package, etc. The app also has a messaging module,
  allowing couriers to communicate among themselves and with a dispatcher. 
  I also developed Spring RESTful web services used by the app. See 
  \href{https://play.google.com/store/apps/details?id=br.com.entregadoronline}{Google Play}} \\
  \multicolumn{2}{c}{} \\

  \textsc{Jan--Jul '13}
  & Programmer \\
  &\emph{College of Engineering \& Computing, University of South Carolina}\\
  &\footnotesize{Developed a web app with some electronic health records (EHR)
  features, managing patient personal data, questionnaires and quizzes, news feed, 
  and more. Google App Engine (Python) was used, along with Bootstrap and jQuery
  on the front end. Under Dr. \href{http://jmvidal.cse.sc.edu}{José Vidal}}\\
  \multicolumn{2}{c}{} \\

  \textsc{Mar '11--Jul '12}
  & Research Assistant \\
  &\emph{Instituto de Informática, Universidade Federal de Goiás}\\
  &\footnotesize{\emph{Concurrent \& Distributed Systems}: Study of models and
  mechanisms of coordination and communication in event-driven concurrent and 
  distributed systems; the use of Lua in the development of higher-level
  concurrency abstractions.}\\
  &\footnotesize{\emph{Smart Cities (IoT)}: Study, experimentation, and Evaluation 
  of IP stack implementations for low-power devices (e.g., TelosB and MicaZ motes).}\\
  &\footnotesize{Both projects under 
    Dr. \href{http://inf.ufg.br/~brunoos/}{Bruno Silvestre}}\\
  \multicolumn{2}{c}{} \\

  \textsc{2011, 2014, 2015}
  & Teaching Assistant \\
  &\emph{Instituto de Informática, Universidade Federal de Goiás} \\
  &\footnotesize{Mar--Jul '15: Formal Languages and Automata} \\
  &\footnotesize{Mar--Jul '14: Operating Systems} \\
  &\footnotesize{Aug--Dec '11: Software Construction}
\end{longtable}

\section{Publications}
\begin{enumerate}
  \renewcommand{\labelenumi}{[\arabic{enumi}] }
  \item Bruno Silvestre, Guilherme Salazar. \emph{Proposta de um Modelo para o
        Processamento de Eventos Concorrentes no ALua.} Encontro Anual de
        Computação, Universidade Federal de Goiás. October 26, 2011
\end{enumerate}

\section{Associations}

\begin{tabular}{r|p{11cm}}

  \textsc{May '16--Cur}
  & Brazilian Computing Society\\
  \multicolumn{2}{c}{}\\

  \textsc{Dec '15--Cur}
  & The NetBSD Foundation
  \\\multicolumn{2}{c}{}\\

  \textsc{Feb '13--Cur}
  & Alpha Lambda Delta Honor Society\\
  \multicolumn{2}{c}{}\\

  \textsc{Dec '11--Cur}
  & Association for Computing Machinery\\

\end{tabular}

\section{Awards}
\begin{tabular}{r|p{11cm}}

  \textsc{2012--2013}
  & Dean's Honor List\\
  & College of Engineering \& Computing, University of South Carolina\\
  \multicolumn{2}{c}{}\\

  \textsc{2012}
  & Study Abroad Scholarship\\
  & Ministry of Education, Government of Brazil\\
  \multicolumn{2}{c}{}\\

  \textsc{2011}
  & Best Paper Award\\
  & Encontro Anual de Computação, Universidade Federal de Goiás\\

\end{tabular}

\section{Skills}

Tools \& technologies I have used and am familiar with

\begin{itemize}
\renewcommand{\labelitemi}{\textcolor{lightg}{\symbol{"00BB}}}
\setlength{\itemsep}{1pt}
\setlength{\parskip}{0pt}
\setlength{\parsep}{0pt}

\begin{multicols}{3}
  \item Portuguese (BR) \hfill \threenotes
  \item English (US)    \hfill \threenotes
\end{multicols}

\begin{multicols}{3}
  \item C          \hfill \threenotes 
  \item Lua        \hfill \threenotes
  \item Python     \hfill \twonotes
  \item Java       \hfill \twonotes
  \item JavaScript \hfill \twonotes
  \item nesC       \hfill \onenote
 \end{multicols}

\begin{multicols}{3}
  \item OpenResty  \hfill \onenote
  \item Flask      \hfill \onenote
  \item GAE        \hfill \twonotes
  \item AWS (EC2)  \hfill \onenote
\end{multicols}

\begin{multicols}{3}
  \item SQL        \hfill \onenote
  \item MySQL      \hfill \onenote
  \item PostgreSQL \hfill \onenote
  \item SQLite     \hfill \onenote
  \item HBase      \hfill \onenote
  \item Redis      \hfill \onenote
\end{multicols}

\begin{multicols}{3}
  \item HTML/CSS   \hfill \twonotes
  \item Bootstrap  \hfill \twonotes
  \item jQuery     \hfill \onenote
  \item Knockout   \hfill \onenote
\end{multicols}

\begin{multicols}{3}
  \item Android   \hfill \twonotes
  \item FreeBSD   \hfill \twonotes
  \item NetBSD    \hfill \twonotes
  \item GNU/Linux \hfill \threenotes
  \item TinyOS    \hfill \onenote
\end{multicols}

\begin{multicols}{3}
  \item Vagrant \hfill \twonotes
  \item Git     \hfill \twonotes
  \item GitHub  \hfill \twonotes
  \item SVN     \hfill \onenote
  \item CVS     \hfill \onenote
\end{multicols}


\end{itemize}

\vspace{1em}

\begin{center}
\parbox[c]{8cm}{
  \onenote Familiar; small-scale projects \\
  \twonotes Quite familiar; used in larger projects \\
  \threenotes Extensive knowledge
}
\end{center}

\end{document}
